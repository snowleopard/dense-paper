\subsection{Dense Graph Representation}\label{sec-dense-graph}

Let us revisit the graph semigroup defined in Section~\ref{subsec:algstr}.
We will denote it by $(G_U, +)$, where $G_U$ is the set of directed graphs whose
vertices come from a universe $U$, that is, if $(V, E) \in G_U$ then
$V \subseteq U$ and $E \subseteq V \times V$. Recall that the graph overlay
operation $+$ is defined as

\[
(V_1, E_1) + (V_2, E_2) = (V_1 \cup V_2, E_1 \cup E_2).
\]

\noindent
The algebra of graphs presented in~\cite{mokhov2017algebraic} also defines
the \emph{graph connect} operation $\rightarrow$:
% \footnote{The definition
% coincides with that of \emph{graph join}~\cite{1969_graph_theory_harary}, but,
% just like graph union, graph join requires that given graphs are
% non-overlapping. The connect operation has no such requirement.}

\[
(V_1, E_1) \rightarrow (V_2, E_2) = (V_1 \cup V_2, E_1 \cup E_2 \cup V_1 \times V_2).
\]

This operation allows us to ``connect'' two graphs by adding edges from every
vertex in the left-hand graph to every vertex in the right-hand graph, possibly
introducing self-loops if $V_1 \cap V_2 \neq \emptyset$. The operation is
associative, non-commutative and distributes over $+$. Note, however, that the
empty graph $\varepsilon = (\emptyset, \emptyset)$ is the identity for both
overlay and connect operations: $\varepsilon + x = x + \varepsilon = x$ and
$\varepsilon \rightarrow x = x \rightarrow \varepsilon = x$, and consequently
the annihilating zero property does not hold, which makes this algebraic
structure not a~semiring according to the classic semiring definition.

By using the two operations one can construct any graph starting from primitive
single-vertex graphs. For example, let $U=\{1,2,3\}$ and $k$ stand for a
single-vertex graph $({k}, \emptyset)$. Then:

\begin{itemize}
  \item $1 \rightarrow 2$ is the graph comprising a single edge $(1,2)$, i.e.
  $1 \rightarrow 2 = (\{1,2\}, \{(1,2)\})$.
  \item $1 \rightarrow (2 + 3)$ is the graph $(\{1,2,3\}, \{(1,2),(1,3)\})$.
  \item $1 \rightarrow 2 \rightarrow 3$ is the graph $(\{1,2,3\}, \{(1,2),(1,3),(2,3)\})$.
\end{itemize}

\noindent
Clearly any sparse graph $(V, E)$, i.e. a graph with a sparse connectivity
matrix, can be constructed by a linear-size expression:

\[
(V, E) = \sum_{v \in V} v + \sum_{(u,v) \in E} u \rightarrow v.
\]

\noindent
But what about complements of sparse graphs, i.e. graphs with dense
connectivity matrices? It is not difficult to show that by applying the dense
linear operator we can obtain a linear-size circuit comprising 2-input gates
$+$ and $\rightarrow$ for any dense graph.

Let $A$ be a dense matrix of size $n\times n$. Our goal is to construct the
graph $G_A = (\{1, \dots, n\}, E)$ such that $(i,j) \in E$ whenever $A_{ij}=1$.

First, we compute the dense linear operator $\mathbf{y} = A \mathbf{x}$ over the
(commutative) graph semigroup~$+$, where $\mathbf{x} = (1, 2, \dots, n)$, i.e.,
$x_j$ is the primitive graph comprising a single vertex~$j$, obtaining
graphs~$y_i$ that comprise sets of isolated vertices corresponding to the rows
of matrix~$A$:

\[
y_i = \sum_{A_{ij}=1} j \, .
\]

The resulting graph $G_A = (\{1, \dots, n\}, E)$ can now be obtained by using
the connect operation~$\rightarrow$ to connect $i$ to all vertices $y_i$, and
subsequently overlaying the results:

\[
G_A = \sum_{i=1}^{n} i \rightarrow y_i.
\]

Thanks to the linear-size construction for the dense linear operator, the size
of the circuit computing $G_A$ is $O(n)$. This allows us to store dense graphs
on $n$ vertices using $O(n)$ memory, and perform basic transformations of dense
graphs in $O(n)$ time. We refer the reader to~\cite{mokhov2017algebraic} for
further details about applications of algebraic graphs in functional programming
languages.
