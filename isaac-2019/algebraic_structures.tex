%!TEX root = main.tex
\subsection{Algebraic Structures}\label{subsec:algstr}

A~\emph{semigroup} $(S, \circ)$ is an algebraic structure, where the operation
$\circ$ is \emph{closed}, i.e., $\circ : S\times S \rightarrow S$, and
\emph{associative}, i.e.,
$x \circ (y \circ z) = (x \circ y) \circ z$ for all $x$, $y$, and $z$ in $S$.
\emph{Commutative} (or \emph{abelian}) semigroups introduce one extra
requirement: $x \circ y = y \circ x$ for all $x$ and $y$ in $S$.

Commutative semigroups are ubiquitous. Below we list a few
notable examples, starting with the most basic one, which is, arguably, known
to every person on the planet.

% Andrey: I agree, but I run out of time :(
% \todo[inline]{Andrey, mne po-prezhnemu ne nravitsya, chto mi tak mnogo mesta (i vremeni chitatelya!) tratim na eti skuchnie algebraicheskie strukturi. U nas pervoe sodergatel'noe dok-vo nachinaetsya na 10-i stranitse tol'ko. Mogesch', pogaluista, bOl'schuyu chast' etoi sektsii unesti v Appendix? Zdes' mogno ostavit' pro $\circ$, $\bullet$ i graph overlay.}

\begin{itemize}
  \item Integer numbers form commutative semigroups with many operations. For
  example, the order in which numbers are \emph{added} is irrelevant, hence
  $(\mathbb{Z}, +)$ is a commutative semigroup. So are $(\mathbb{Z}, \times)$,
  $(\mathbb{Z}, \min)$ and $(\mathbb{Z}, \max)$. On the other hand, it does
  matter in which order numbers are \emph{subtracted}, hence $(\mathbb{Z}, -)$
  is not a commutative semigroup: $1-2 \neq 2-1$. In fact, $(\mathbb{Z}, -)$
  is not even a semigroup, since subtraction is non-associative:
  $1-(2-3) \neq (1-2)-3$.

  \item Boolean values form commutative semigroups $(\mathbb{B}, \vee)$,
  $(\mathbb{B}, \wedge)$, $(\mathbb{B}, \oplus)$ and $(\mathbb{B}, \equiv)$.

  \item Any commutative semigroup $(S, \circ)$ can be \emph{lifted} to the set
  $\hat{S}$ of ``containers'' of elements $S$, e.g., vectors or matrices,
  obtaining a commutative semigroup $(\hat{S}, \hat{\circ})$, where the lifted
  operation~$\hat{\circ}$ is applied to the contents of containers element-wise.
  % The \cdot in \hat{\cdot} is a placeholder for an operation that is lifted
  The lifting operation $\hat{\cdot}$ can often be omitted for clarity if there
  is no ambiguity.

  The \emph{average semigroup} $(\mathbb{Z} \times \mathbb{Z}, \circ)$ is a
  simple yet not entirely trivial example of semigroup lifting. By defining
  $(t_1, c_1) \circ (t_2, c_2) = (t_1 + t_2, c_1 + c_2)$, we can aggregate
  partial \emph{totals} and \emph{counts} of a set of numbers, which allows us
  to efficiently calculate their average as $\textit{avg}(t, c) = \frac{t}{c}$.
  The average semigroup is commutative.

  \item Set union and intersection are commutative and associative operations
  giving rise to many set-based commutative semigroups. Here we highlight an
  example that motivated our research: the \emph{graph overlay} operation,
  defined\footnote{This definition coincides with that of the
  \emph{graph union} operation~\cite{1969_graph_theory_harary}. Graph union
  typically requires that given graphs are non-overlapping, hence it is not
  closed on the set of all graphs. Graph overlay does not have such a
  requirement, and is therefore closed and forms a semigroup.} as
  $(V_1, E_1) + (V_2, E_2) = (V_1 \cup V_2, E_1 \cup E_2)$,~where~$(V, E)$ is
  a standard set-based representation for directed unweighted graphs, comes from
  an algebra of graphs used in functional programming~\cite{mokhov2017algebraic}.
  See further details in Section~\ref{sec-dense-graph}.
  % We will use the commutative semigroup $(G_U,+)$, where $G_U$ are graphs whose
  % vertices come from the universe $U$, for finding compact algebraic
  % representations for \emph{dense graphs}, i.e. complements of sparse graphs,
  % in Section~\ref{sec-dense-graph}.
\end{itemize}

\emph{Groups} extend semigroups by requiring the existence of the \emph{identity
element} $0 \in S$, such that $0 \circ x = x \circ 0=x$, and the \emph{inverse
element} $-x \in S$ for all $x \in S$, such that
$(-x) \circ x = x \circ (-x) = 0$. Groups provide a natural generalisation of
arithmetic \emph{subtraction}, whereby $x \circ (-y)$ denotes the subtraction of
$y$ from $x$.

A commutative semigroup $(S, \circ)$ can often be extended to a \emph{semiring}
$(S, \circ, \bullet)$ by introducing another associative (but not necessarily
commutative) operation $\bullet$ that \emph{distributes} over $\circ$, that is
\[
x \bullet (y \circ z) = (x \bullet y) \circ (x \bullet z)\\
\]
\[
(x \circ y) \bullet z = (x \bullet z) \circ (y \bullet z)
\]
hold for all $x$, $y$, and $z$ in~$S$. Since $\circ$ and $\bullet$~behave
similarly to numeric addition and multiplication, it is common to give $\bullet$
a higher precedence to avoid unnecessary parentheses, and even omit~$\bullet$~from
formulas altogether, replacing it by juxtaposition. This gives a terser and
more convenient notation, e.g., the left distributivity law becomes:
$x (y \circ z) = x y \circ x z$. We will use this notation, insofar as this does
not lead to ambiguity.

Most definitions of semirings also require that the two operations have
identities: the \emph{additive identity}, denoted by 0, such that
$0 \circ x = x \circ 0=x$, and the \emph{multiplicative identity}, denoted by 1,
such that $1x=x1=x$. Furthermore, 0 is typically required to be
\emph{annihilating}: $0x=x0=0$.

A semiring $(S, \circ, \bullet)$ is also a \emph{ring} if $(S, \circ)$ is a
group, i.e., the operation $\circ$ is invertible. One can think of rings as
semirings with subtraction.

Let us extend some of our semigroup examples to semirings:

\begin{itemize}
  \item The most basic and widely known semiring is that of integer numbers with
  addition and multiplication: $(\mathbb{Z}, +, \times)$. Since every integer
  number $x\in \mathbb{Z}$ has an inverse $-x \in \mathbb{Z}$ with respect to
  the addition operation, $(\mathbb{Z}, +, \times)$ is also a ring.
  Interestingly, integer addition can also play the role of multiplication when
  combined with the $\max$ operation, resulting in the \emph{tropical semiring}
  $(\mathbb{Z}, \max, +)$, which is also known as the \emph{max-plus algebra}.
  Unlike $+$, the $\max$ operation has no inverse, therefore
  $(\mathbb{Z}, \max, +)$ is not a ring.

  \item Boolean values form the semiring $(\mathbb{B}, \vee, \wedge)$. Note that
  $(\mathbb{B}, \wedge, \vee)$ is a semiring too thanks to the duality between
  the operations $\vee$ and $\wedge$. Furthermore,
  $(\mathbb{B}, \oplus, \wedge)$ is a ring, where every element is its own
  inverse: $x \oplus x = 0$ for $x \in \mathbb{B}$.

  \item Semirings and rings $(S, \circ, \bullet)$ can also be lifted to the set
  $\hat{S}$ of ``containers'' of elements $S$, most commonly matrices, obtaining
  $(\hat{S}, \hat{\circ}, \hat{\bullet})$. Matrices over tropical semirings, for
  example, are used for solving various path-finding problems on graphs.
\end{itemize}

% \todo[inline]{Andrey, Chazelle and Rosenberg also mention powerset as an
% idempotent and commutative semigroup. Should we also mention it? AM: Yes, I
% think we can add it too. I wonder if we could also come up with a good
% application based on powerset semirings. Furthermore, if in addition to
% commutativity we also have idempotence, do we gain anything? Could we get a
% smaller constant in our linear construction?}
