% !TEX root = main.tex

\subsection{Explicit Construction for Commutative Case}\label{subsec:constructive}

Our original proof was probabilistic in the part computing the matrix $R_1 \times C$ having $n$ columns, at most $n$ rows, at most $k$ zeros and at most $\log n$ zeros in each row. For our proof it is convenient to add redundant rows and columns to the matrix in such a way that it has size $t \times t$, at most $t$ zeros in total and at most $\log t$ zeros in each row. We will then compute the matrix by the circuit of size at most $O(t)$. To enlarge the matrix we can just add rows and columns consisting of ones. It is easy to see that then we can let $t = \max(n,k)$. Note that enlargement of the matrix does not make the computation simpler: additional rows mean additional outputs that can be ignored and additional columns correspond to redundant variables that can be removed (substituted by 0) once the circuit is constructed.

Thus our goal is to compute matrix $A \in \{0,1\}^{t\times t}$ with at most $t$ zeros in total and at most $\log t$ zeros in each row by a circuit of size $O(t)$. For this we will first permute the columns of $A$ in such a way that all ranges of length shorter than $\log t$ lie within the last $O(\log^2 t)$ columns. Once we have such an permutation of columns, we can compute ranges of length at least $\log t$ by Lemma~\ref{lemma:blocks} and we can compute all ranges within the last $O(\log^2 n)$ columns by a circuit of size at most $O(\log^4 n)$, thus computing all short ranges. 

Thus it remains to construct a desired permutation of $A$. We will do it step by step by a greedy algorithm. After step $r$ we will have a sequence of the first $r$ columns chosen and we will maintain the following properties:
\begin{itemize}
\item For each $i \leq r$ the first $i$ columns contain at least $i$ zeros;
\item There are no ranges of the length less than $\log n$ in the first $r$ rows (apart from those, that can be extended by adding columns on the right).
\end{itemize}
The process will work for at least $t - \log^ t$ steps, so the ranges of length $\log t$ are only possible within the last $\log^2 t + \log t = O(\log^2 t)$ columns.

On the first step we can pick for the first column any column that has zeros in it. Suppose we have reached step $r$. We will now explain how to add a column on step $r+1$. Consider the last $\log t$ columns in the currently constructed list. Consider the set $A$ of rows that have zeros in them. These are exactly the rows that are placing constraints on the choice of our next column. There are two cases.
\begin{enumerate}
\item There are at most $\log t$ rows in $A$. Then, for each row in $A$ there are at most $\log t$ columns that has zeros in this row. In total, there are at most $\log^2 t$ columns that have zeros in some rows of $A$. Denote the set of this columns by $F$. If there is an unpicked column outside of $F$ that has at least one zero in it, we add this column to our sequence. Clearly, the properties of our sequence are maintained and the step is over. Otherwise, all other columns contain only ones, so we add all of them to our sequence, place the columns from $F$ to the end of the sequence, and the whole permutation is constructed.
\item There are more that $\log t$ rows in $A$. This means that in the last $\log t$ columns of the current sequence there are more than $\log t$ zeros. Also note, that by the maintained property, the first $r - \log t$ columns there are at least $r - \log t$ zeros. So overall, in the current sequence of $r$ columns there are more than $r$ zeros. Thus, in the remaining $t-r$ columns there are less then $t-r$ zeros and there is a column without zeros. Let us add this column to our sequence. It is easy to see that the properties are still maintained.
\end{enumerate}