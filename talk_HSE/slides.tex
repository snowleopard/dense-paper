\documentclass{beamer}
%\includeonlyframes{ew}
\usepackage{graphicx, amssymb}
\usepackage{amsmath}
\usepackage{tikz}
\usepackage{kbordermatrix}

\usetikzlibrary{shapes.arrows,chains,decorations.pathreplacing}
\usetikzlibrary{arrows, decorations.markings, shapes}
\usetikzlibrary{positioning}

\usepackage{theory}



%\usepackage{easybmat}
\usepackage{blkarray}

\newtheorem{hypothesis}{Hypothesis}
%\newcommand{\poly}{\text{poly}}

\setbeamertemplate{footline}[frame number]
\setbeamertemplate{navigation symbols}{}

%\usepackage[T2A]{fontenc}
%\usepackage[cp1251]{inputenc}
%%\usepackage[koi8-r]{inputenc}
%\usepackage[russian]{babel}

\colorlet{dark}{red!85!blue!60!black}
\newcommand{\emp}[1]{{\bf\color{dark}#1}}

\newtheorem{observation}[theorem]{Observation}
%\newtheorem{problem}[theorem]{Open Problem}

\setbeamertemplate{caption}{\raggedright\insertcaption\par}

\def\mpfile#1#2{\includegraphics{#1-#2.mps}}

\author{Vladimir V. Podolskii\inst{1}\\[3mm] \small joint work with Alexander Kulikov, Ivan Mikhailin, Andrey Mokhov}



\institute{
 \inst{1} \  Steklov\ Mathematical\ Institute, Moscow\\
 Higher School of Economics, Moscow\\
}

\date{HSE Open Lectures}

\title{Complexity of Dense Linear Operators}

\newcommand{\newfontobj}[2]{
  \newcommand{#1}[1]{
    \expandafter\def\csname##1\endcsname{\ensuremath{#2{##1}}}}}

%\newfontobj{\class}{\mathrm}
%\class{P} \class{FP} \class{UP} \class{FewP}
%\class{NP}\class{coNP} \class{SAT} \class{coNP} \class{PSPACE}
%\class{EXP} \class{QBF} \class{TFNP} \class{TF} \class{F}
%\class{PH} \class{AND} \class{XOR} \class{PARITY} \class{OR}
%\class{MAJ} \class{GT} \class{PARITY}

\newcommand{\PRIMES}{\textsc{PRIMES}}

\newcommand{\bb}[1]{\mathbb{#1}}
%\newcommand{\zo}{\{0,1\}}

\newcommand{\vertex}{\textsc{VertexCover}}
\newcommand{\tropdim}{\textsc{TropDim}}
\newcommand{\tropsolv}{\textsc{TropSolv}}
\newcommand{\tropimpl}{\textsc{TropImpl}}
\newcommand{\tropequiv}{\textsc{TropEquiv}}

\newcommand{\para}[1]{{\color{blue}#1}}

\begin{document}

\tikzstyle{mbrace} = [decorate,decoration={brace,amplitude=5pt}]


\begin{frame}

\titlepage

\end{frame}



\begin{frame}
\frametitle{Setting}

Consider a Boolean matrix $A \in \{0,1\}^{n\times n}$\\
Consider variables $x = (x_1,\ldots, x_n)$ over $\{0,1\}$

\medskip
We want to compute a Boolean linear operator $A x$

\medskip

\[
  A =\begin{bmatrix}
    1 & 0 & 1 & 0 \\
    1 & 1 & 0 & 0 \\
    0 & 1 & 1 & 1 \\
    1 & 1 & 1 & 1 \\
  \end{bmatrix}
 \ \ \ \ \ \ \ \ \ \ \ \ \ \ \ \ Ax = \left(\begin{array}{@{}l@{}}
    x_1 \vee x_3 \\
    x_1 \vee x_2 \\
    x_2 \vee x_3 \vee x_4 \\
    x_1 \vee x_2 \vee x_3 \vee x_4\\
  \end{array}\right)
  \]



\end{frame}



\begin{frame}
\frametitle{Model}

\begin{itemize}
\item The computation is a Boolean circuit consisting of $\OR$ gates
\item We start with variables $x_1,\ldots, x_n$
\item In one step we can compute OR of two previously computed expressions
\item Want to compute all the outputs and minimize the number of steps
\end{itemize}

\end{frame}


\begin{frame}
\frametitle{Example}

\[
  A =\begin{bmatrix}
    1 & 0 & 1 & 0 \\
    1 & 1 & 0 & 0 \\
    0 & 1 & 1 & 1 \\
    1 & 1 & 1 & 1 \\
  \end{bmatrix}
 \ \ \ \ \ \ \ \ \ \ \ \ \ \ \ \ \ \ \ \ \begin{array}{l}
    x_1 \vee x_3 \\
    x_1 \vee x_2 \\
    x_2 \vee x_3 \vee x_4 \\
    x_1 \vee x_2 \vee x_3 \vee x_4\\
  \end{array}
  \]

\medskip
Computation:

$x_1 \vee x_3$ --- output\\
$x_1 \vee x_2$ --- output\\
$x_3 \vee x_4$\\
$x_2 \vee (x_3 \vee x_4)$ --- output\\
$x_1 \vee (x_2 \vee x_3 \vee x_4)$ --- output

\end{frame}


\begin{frame}
\frametitle{Basic facts}

\begin{itemize}%[<+->]
\item One of the simplest Boolean circuit complexity models, studied since 50's
\item Trivial upper bound: $O(n^2)$
\item Counting lower bound: $\Omega(n^2/\log n)$
\item Non-trivial upper bound: $O(n^2/\log n)$ (Lupanov '56)
\item The best explicit lower bound: $\Omega(n^{2 - o(1)})$ (Nechiporuk '70)
\end{itemize}



\end{frame}


\begin{frame}
\frametitle{General setting}

Consider a Boolean matrix $A \in \{0,1\}^{n\times n}$\\
Consider variables $x = (x_1,\ldots, x_n)$ over some semigroup $(S, \circ)$.

\medskip
We want to compute a linear operator $A x$.

\medskip

\[
  A =\begin{bmatrix}
    1 & 0 & 1 & 0 \\
    1 & 1 & 0 & 0 \\
    0 & 1 & 1 & 1 \\
    1 & 1 & 1 & 1 \\
  \end{bmatrix}
 \ \ \ \ \ \ \ \ \ \ \ \ \ \ \ \ \ \ \ \ Ax = \left(\begin{array}{@{}l@{}}
    x_1 \circ x_3 \\
    x_1 \circ x_2 \\
    x_2 \circ x_3 \circ x_4 \\
    x_1 \circ x_2 \circ x_3 \circ x_4\\
  \end{array}\right)
  \]

\end{frame}


\begin{frame}
\frametitle{Some semigroups}

\begin{itemize}%[<+->]
\item Boolean semigroup: $(\{0,1\}, \vee)$
\item Integers with addition: $(\mathbb{Z}, +)$
\item $\{0,1\}$ with addition: $(\{0,1\}, \oplus)$
\item Tropical semigroup: $(\mathbb{Z}, \min)$

\end{itemize}


\end{frame}


\begin{frame}
\frametitle{The Problem}

\para{Problem:} Suppose $A \in \{0,1\}^{n\times n}$ is very dense, that is $A$ has $O(n)$ zeros. How hard is it to compute $Ax$?

\pause
\medskip
Simple observations:
\begin{itemize}[<+->]
\item If instead we consider $A$ containing $O(n)$ ones, the complexity is trivially $O(n)$
\item If $(S,\circ)$ has an inverse operation (is a group), the complexity is trivially $O(n)$
\end{itemize}



\end{frame}


\begin{frame}
\frametitle{Motivation}

\begin{itemize}[<+->]
\item Want to understand the structure of semigroups that are not groups
\item Important examples: Boolean semigroup and tropical semigroup
\item Famous problem of similar flavor: matrix multiplication \\
Given two matrices $A$, $B$, how many operations are needed to compute $A\cdot B$?
\item Non-trivial upper bound over integers: $O(n^{2.373})$ (V.~Williams '12)
\item Best known upper bound over tropical semiring is $O(n^3/e^{\sqrt{\log n}})$ (R.~Williams '14)
\item Other motivation: connection to range minimum query problem (will see later)
\end{itemize}


\end{frame}


\begin{frame}
\frametitle{Main results}

\begin{theorem}
If $(S,\circ)$ is a commutative semigroup, then $Ax$ can be computed in $O(n)$ operations for dense $A$
\end{theorem}

\pause

\begin{theorem}
If $(S,\circ)$ is strongly non-commutative semigroup, then the maximal complexity of $Ax$ is $\Theta(n\alpha(n))$ operations for dense $A$
\end{theorem}

\medskip
Here $\alpha(n)$ is the inverse Ackermann function

\pause
\medskip
$\alpha(n)$ growth is extremely slow

\medskip
For all practical needs we can assume $\alpha(n)\leq 4$

\end{frame}



\begin{frame}
\frametitle{Plan}

\begin{itemize}
\item Upper bound + connection to RMQ
\item A bit on lower bounds (+ connection to RMQ)
\end{itemize}



\end{frame}



\begin{frame}
\frametitle{Upper bound}

\begin{theorem}[restated]
If $(S,\circ)$ is a commutative semiring, then $Ax$ can be computed is $\Theta(n)$ for dense $A$
\end{theorem}


\begin{itemize}
\item Let's concentrate on Boolean case: $(\{0,1\},\vee)$
\item The general case is the same
\item So, $A \in \{0,1\}^{n\times n}$ has $O(n)$ zeros, we want to compute $Ax$ using $O(n)$ operations
\end{itemize}


\end{frame}



\begin{frame}
\frametitle{Warm Up}

Consider
\[
 A =\begin{bmatrix}
    0 & 1 & 1 & 1 & 1 & 1 \\
    1 & 0 & 1 & 1 & 1 & 1 \\
    1 & 1 & 0 & 1 & 1 & 1 \\
    1 & 1 & 1 & 0 & 1 & 1 \\
    1 & 1 & 1 & 1 & 0 & 1 \\
    1 & 1 & 1 & 1 & 1 & 0 \\
    \end{bmatrix}
\]

How would we compute $Ax$?

\pause
\medskip
Compute:
\vspace{-3mm}
\begin{equation*}
\begin{array}{l}
\\
\text{all prefixes}\\
x_1\\
x_1 \vee x_2\\
x_1 \vee x_2 \vee x_3\\
x_1 \vee x_2 \vee x_3 \vee x_4\\
x_1 \vee x_2 \vee x_3 \vee x_4 \vee x_5\\
\end{array}\ 
\only<4->{\begin{array}{c}
\text{match}\\
\\
\leftrightarrow\\
\leftrightarrow\\
\leftrightarrow\\
\leftrightarrow\\
\\
\end{array}\ }
\only<-3>{\phantom{\begin{array}{c}
\text{match}\\
\\
\leftrightarrow\\
\leftrightarrow\\
\leftrightarrow\\
\leftrightarrow\\
\\
\end{array}\ }}
\only<3->{\begin{array}{r}
\\
x_2 \vee x_3 \vee x_4 \vee x_5 \vee x_6\\
x_3 \vee x_4 \vee x_5 \vee x_6\\
x_4 \vee x_5 \vee x_6\\
x_5 \vee x_6\\
x_6\\
\text{all suffixes}\\
\end{array}}
\only<-2>{\phantom{\begin{array}{r}
\\
x_2 \vee x_3 \vee x_4 \vee x_5 \vee x_6\\
x_3 \vee x_4 \vee x_5 \vee x_6\\
x_4 \vee x_5 \vee x_6\\
x_5 \vee x_6\\
x_6\\
\text{all suffixes}\\
\end{array}}}
\end{equation*}

\end{frame}



\begin{frame}
\frametitle{Warm Up-2}

Suppose $A \in \{0,1\}^{n\times n}$ has 2 zeros in each row. How would we compute $Ax$?

\pause
\medskip
Permute columns of $A$ randomly and cut in two halves.

\begin{equation*}\only<-2>{\left[
\begin{array}{@{}ccc|ccc@{}}
     & 0 &  &  & 0 &  \\
    0 &  &  &  &  & 0 \\
     &  & 0 & 0 &  &  \\
     &  &  & 0 & 0 &  \\
     &  &  &  & 0 & 0 \\
     0 & 0 &  &  &  &  \\
    \end{array}
    \right]}
\only<3->{
\left[
\begin{array}{@{}ccc|ccc@{}}
     & \color{red}0 &  &  & \color{red}0 &  \\
    \color{red}0 &  &  &  &  & \color{red}0 \\
     &  & \color{red}0 & \color{red}0 &  &  \\\hline
     &  &  & 0 & 0 &  \\
     &  &  &  & 0 & 0 \\
     0 & 0 &  &  &  &  \\
    \end{array}
    \right]
}
\end{equation*}

\begin{overlayarea}{110mm}{25mm}
\only<3>{
With positive probability in at least half of the rows the zeros will be splitted. 

Compute them by the previous algorithm, compute the rest recursively.
}
\only<4->{
We get the recurrence
$$
T(n) = Cn + T(n/2),
$$
where $T(n)$ is the complexity for matrices with $n$ rows
}

\end{overlayarea}



\end{frame}


\begin{frame}
\frametitle{Towards the General Case}

How would we solve the general case?

\medskip
First idea: Connection to Range Minimum Query problem (RMQ)

\medskip
This is the standard setting in theory of algorithms

\medskip
We are given an array of numbers $x_1,\ldots, x_n$. We want a data structure to answer queries of the form 
$$
\min\{x_i \mid l\leq i \leq r\}=\ ?
$$ 
for integer $l$ and $r$.

\end{frame}


\begin{frame}
\frametitle{Reduction to RMQ}

Consider
\[
 A =\begin{bmatrix}
    1 & 0 & 1 & 1 & 0 & 1 \\
    1 & 1 & 0 & 1 & 1 & 1 \\
    1 & 0 & 1 & 1 & 1 & 0 \\
    1 & 1 & 1 & 0 & 1 & 1 \\
    0 & 1 & 1 & 1 & 1 & 1 \\
    1 & 0 & 0 & 1 & 1 & 1 \\
    \end{bmatrix}
\]

Split each row into intervals
\[
\begin{array}{lll}
    x_1 & \phantom{0}x_3 \vee x_4 & \phantom{0}x_6 \\
    x_1 \vee x_2& \phantom{0}x_4\vee x_5 \vee x_6 & \\
    x_1& \phantom{0}x_3\vee x_4 \vee x_5 &\\
    x_1\vee x_2 \vee x_3& \phantom{0}x_5 \vee x_6 &\\
    x_2\vee x_3 \vee x_4\vee x_5 \vee x_6 &&\\
    x_1& \phantom{0}x_4 \vee x_5\vee x_6 &\\
    \end{array}
\]

There are $O(n)$ intervals in total, so we reduced our problem to the offline version of RMQ (intervals are given in advance)


\end{frame}


\begin{frame}
\frametitle{Complexity of RMQ}

Unfortunately, best constructions for RMQ give only $O(n \alpha(n))$ complexity in our model, where $\alpha(n)$ is an inverse Ackermann function


\pause
\medskip
Moreover, the following is true
\begin{theorem}[Chazelle, Rozenberg '91]
There are range matrices $A\in \{0,1\}^{n\times n}$ with the complexity $\Omega(n \alpha(n))$
\end{theorem}

So, the reduction to RMQ is not enough


\end{frame}



\begin{frame}
\frametitle{Idea}


\begin{equation*}\only<-2>{A=\left[
\begin{array}{@{}cccccc@{}}
     \ast & \ast  & \ast  & \ast  & \ast  & \ast  \\
     \ast & \ast & \ast  & \ast  & \ast  & \ast  \\
     \ast & \ast & \ast  & \ast  & \ast  & \ast  \\
     \ast & \ast & \ast  & \ast  & \ast  & \ast  \\
     \ast & \ast & \ast  & \ast  & \ast  & \ast  \\
     \ast & \ast & \ast  & \ast  & \ast  & \ast  \\
    \end{array}
    \right]}
\only<3->{
A=\left[
\begin{array}{@{}ccc|ccc@{}}
     \ast & \ast  & \ast  & \ast  & \ast  & \ast  \\
     \ast & \ast & \ast  & \ast  & \ast  & \ast  \\
     \ast & \ast & \ast  & \ast  & \ast  & \ast  \\
     \ast & \ast & \ast  & \ast  & \ast  & \ast  \\
     \ast & \ast & \ast  & \ast  & \ast  & \ast  \\
     \ast & \ast & \ast  & \ast  & \ast  & \ast  \\
    \end{array}
    \right]
}
\end{equation*}

There is a simple construction with complexity $O(n \log n)$

\pause

\begin{itemize}[<+->]
\item Compute all prefixes and suffixes
\item Split the matrix in two halves vertically
\item Repeat in both halves recursively
\item Now we can compute any range in just one operation, just pick the first vertical line that crosses it
\end{itemize}

\end{frame}


\begin{frame}
\frametitle{Next Idea}

\begin{lemma}
Suppose $A \in \{0,1\}^{n\times n}$ has $O(n)$ zeros and each range is of length at least $\log n$. Then we can compute $Ax$ in $O(n)$ operations
\end{lemma}

\begin{itemize}
\item Split into blocks of size $\log n$
\item Each range intersects block boundary!
\item Compute prefixes and suffixes in each block in $O(n)$ operations
\item Apply the previous idea on top of blocks
\item Now for each range we can compute its `complete' part in one operation and add `incomplete' prefix and suffix in two more operations
\end{itemize}

\end{frame}

\begin{frame}
\frametitle{Yet Another Next Idea}

\begin{lemma}
Suppose $A \in \{0,1\}^{n\times n}$ has $O(n)$ zeros and at most $\log n$ zeros in each row. Then we can compute it in $O(n)$ operations
\end{lemma}

\begin{itemize}
\item Permute blocks randomly
\item With high probability most zeros will be far from each other
\item Thus most ranges are long
\item Compute them by the previous idea
\item Compute all short ranges by brute force (there are few of them)
\end{itemize}

\end{frame}


\begin{frame}
\frametitle{Final Idea}

\begin{theorem}
Suppose $A \in \{0,1\}^{n\times n}$ has complexity $s$. Then $A^T$ also has complexity $s$
\end{theorem}


\pause
\begin{center}
\begin{tikzpicture}
%\draw[help lines] (0,0) grid (10,6);
\foreach \x/\y/\n/\t in {0/4/x1/1, 1/4/x2/2, 2/4/x3/3, 3/4/x4/4, 4/4/x5/5, 2/3/a/~, 1/2/b/1, 3/2/c/2, 4/2/d/3}
  \node[inner sep=0mm,circle,draw,minimum size=6mm] (\n) at (\x,\y) {$\t$};
\foreach \s/\t in {x2/a, x3/a, x4/a, x1/b, a/b, x5/c, a/c, x4/d, x5/d}
  \draw[->] (\s) -- (\t);

\node at (8,3) {$A=\begin{bmatrix}1&1&1&1&0\\0&1&1&1&1\\0&0&0&1&1\end{bmatrix}$};
\end{tikzpicture}
\end{center}

\begin{center}
\begin{tikzpicture}
\foreach \x/\y/\n/\t in {0/4/x1/1, 1/4/x2/2, 2/4/x3/3, 3/4/x4/4, 4/4/x5/5, 2/3/a/~, 1/2/b/1, 3/2/c/2, 4/2/d/3}
  \node[inner sep=0mm,circle,draw,minimum size=6mm] (\n) at (\x,\y) {$\t$};
\foreach \s/\t in {x2/a, x3/a, x4/a, x1/b, a/b, x5/c, a/c, x4/d, x5/d}
  \draw[<-] (\s) -- (\t);

\node at (8,3) {$A^T=\begin{bmatrix}1&0&0\\1&1&0\\1&1&0\\1&1&1\\0&1&1\end{bmatrix}$};
\end{tikzpicture}
\end{center}


\end{frame}


\begin{frame}
\frametitle{So, What Do We Have?}

\begin{itemize}
\item Can compute $Ax$ in $O(n\log n)$ operations
\item If each row has at most $\log n$ zeros, can compute $Ax$ in $O(n)$ opetations
\item $Ax$ and $A^Ty$ have the same complexity
\end{itemize}


\end{frame}

\begin{frame}
\frametitle{Completing the Proof}

\begin{equation*}\only<-1>{A=\left[
\begin{array}{@{}cccccc@{}}
     \ast & \ast  & \ast  & \ast  & \ast  & \ast  \\
     \ast & \ast & \ast  & \ast  & \ast  & \ast  \\
     \ast & \ast & \ast  & \ast  & \ast  & \ast  \\
     \ast & \ast & \ast  & \ast  & \ast  & \ast  \\
     \ast & \ast & \ast  & \ast  & \ast  & \ast  \\
     \ast & \ast & \ast  & \ast  & \ast  & \ast  \\
    \end{array}
    \right]}
\only<2->{
A=\left[
\begin{array}{@{}cccccc@{}}
     \ast & \ast  & \ast  & \ast  & \ast  & \ast  \\
     \ast & \ast & \ast  & \ast  & \ast  & \ast  \\
     \hline
     \ast & \ast & \ast  & \ast  & \ast  & \ast  \\
     \ast & \ast & \ast  & \ast  & \ast  & \ast  \\
     \ast & \ast & \ast  & \ast  & \ast  & \ast  \\
     \ast & \ast & \ast  & \ast  & \ast  & \ast  \\
    \end{array}
    \right]
}
\end{equation*}

Suppose we have $A$ with $O(n)$ zeros

\pause
\begin{itemize}[<+->]
\item Split rows in two parts: with more than $\log n$ zeros and with at most $\log n$ zeros
\item We know how to compute the second part
\item The first part has at most $n/\log n$ rows
\item Transpose the first part
\item The transposition is computable in $O(n)$ operations
\end{itemize}


\end{frame}

\begin{frame}
\frametitle{Non-commutative Case}

Recall, that we heavily used commutativity even for the case when $A$ has two zeros in each row: permutation of columns

\pause
\medskip
We next show that this is unavoidable:

\medskip
\begin{theorem}
If $(S,\circ)$ is strongly non-commutative semiring, then there is $A$ with at most 2 zeros in each row such that the complexity of $Ax$ is $\Omega(n\alpha(n))$
\end{theorem}

\end{frame}

\begin{frame}
\frametitle{Examples}

\begin{itemize}[<+->]
\item Recall Boolean case. There are equivalences on variables:
$$
x_1 \vee x_1 = x_1, \ \ \ \ x_1 \vee x_2 = x_2 \vee x_1
$$
\item The first is idempotency, the second is commutativity
\item The first example for us is free semigroup: no equivalences on variables
\item The second example is free idempotent semigroup; the only equivalences are
$$
x \circ x = x
$$
\end{itemize}



\end{frame}

\begin{frame}
\frametitle{Idempotent Case}

We prove the following problem equivalences 

\begin{center}
\begin{tikzpicture}[xscale=0.65,line width=1pt]
%\draw[help lines] (0,0) grid (16,6);
\tikzstyle{v}=[rectangle,draw,inner sep=1mm,text width=26mm,above right,minimum height=25mm]

\node[v] (a) at (0,0) {commutative RMQ has $O(s)$ complexity};

\node[v] (b) at (6.5,0) {non-commutative RMQ has $O(s)$ complexity};

\node[v] (c) at (13,0) {non-commutative dense operators $Ax$ have $O(s)$ complexity};

\path (a.10) edge[->] node[above=1cm] {complicated} (b.170);
\path (b.190) edge[->] node[below=1cm] {special case} (a.-10);
\path (b.10) edge[->] node[above=1cm] {straightforward} (c.170);
\path (c.190) edge[->] node[below=1cm] {complicated} (b.-10);
%\node (l1) at (6.5,0) {special case};
\end{tikzpicture}
\end{center}



\end{frame}

\begin{frame}
\frametitle{Non-idempotent Case}

\begin{itemize}
\item Suppose we have a small circuit
\item Just factorize the whole computation by idempotency relations
\item Now we have a small circuit for idempotent case
\item This is a contradiction
\item Basically, non-idempotent case is harder
\end{itemize}



\end{frame}

\begin{frame}
\frametitle{Open problem}

\begin{problem}[Jukna '19]
Consider $A \in \{0,1\}^{n\times n}$, denote by $\overline{A}$ the bit-wise negation of $A$. How large can
$$
\frac{Complexity(\overline{A}x)}{Complexity(Ax)}
$$ 
be (over Boolean semigroup)?
\end{problem}

\pause
\medskip
Our result rules out the possibility to achieve gap with sparse matrix

\pause
\medskip
Thank you for attention!

\end{frame}

\end{document}
